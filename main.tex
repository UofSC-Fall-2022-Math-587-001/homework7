\documentclass[12pt]{amsart}
\usepackage{amsmath}
\usepackage{amsthm}
\usepackage{amsfonts}
\usepackage{amssymb}
\usepackage[margin=1in]{geometry}
\usepackage{hyperref}
\hypersetup{
    colorlinks=true,
    linkcolor=blue
}

\theoremstyle{definition}
\newtheorem{theorem}{Theorem}[section]
\newtheorem{lemma}[theorem]{Lemma}
\newtheorem{definition}[theorem]{Definition}
\newtheorem{corollary}[theorem]{Corollary}
\newtheorem{proposition}[theorem]{Proposition}
\newtheorem{conjecture}[theorem]{Conjecture}
\newtheorem{remark}[theorem]{Remark}
\newtheorem{example}[theorem]{Example}
\newtheorem{problem}[theorem]{Problem}
\newtheorem{notation}[theorem]{Notation}
\newtheorem{question}[theorem]{Question}
\newtheorem{caution}[theorem]{Caution}

\begin{document}

\title{Homework 6}

\maketitle

For this week, please answer the following questions from the text. 
I've copied the problem itself below and the question numbers for 
your convenience. 

\begin{enumerate}
	\item (2.18) Solve each of the following simultaneous systems of 
		congruences (or explain why no solution exists). 
	\begin{enumerate}
		\item $x = 3 \mod 7$ and $x = 4 \mod 9$
		\item $x = 137 \mod 423$ and $x = 87 \mod 191$
		\item $x = 133 \mod 451$ and $x = 237 \mod 697$
		\item $x = 5 \mod 9$, $x = 6 \mod 10$, and $x = 7 \mod 11$
		\item $x = 37 \mod 43$, and $x = 22 \mod 49$, and $x = 
			18 \mod 71$. 
	\end{enumerate}
	\item (2.21) 
	\begin{enumerate}
		\item Let $a,b,c$ be positive integers and suppose that 
		\begin{displaymath}
			a \mid c, b \mid c, \operatorname{and} 
			\operatorname{gcd}(a,b) = 1
		\end{displaymath}
		Prove that $ab \mid c$.
		\item Let $c$ and $c^\prime$ be two solutions to the system of 
			simultaneous congruences (2.7) in the Chinese 
			remainder theorem (Theorem 2.24). Prove that 
		\begin{displaymath}
			c = c^\prime \mod m_1 m_2 \cdots m_k 
		\end{displaymath}
			
	\end{enumerate}
		
	\item (2.23) Use the method described in Sect. 2.8.1 to find square 
		roots modulo the following composite moduli.
	\begin{enumerate}
		\item Find a square root of $340$ modulo $437$. (Note that 
			$437 = 19 \cdot 23$.) 
		\item Find a square root of $253$ modulo $3143$. 
		\item Find four square roots of $2833$ modulo $4189$. (The 
			modulus factors as $4189 = 59\cdot 71$. Note that your 
			four square roots should be distinct modulo $4189$.)
		\item Find eight square roots of $813$ modulo $868$. 
	\end{enumerate}
	\item (2.25) Suppose $n=pq$ with $p$ and $q$ distinct odd primes. 
	\begin{enumerate}
		\item Suppose that $\operatorname{gcd}(a,pq) = 1$. Prove that 
			if the equation $x^2 = a \mod n$ has any solution 
			then it has four solutions. 
		\item Suppose that you have a machine that could find all four 
			solutions for some $a$. How could you use this machine 
			to factor $n$?
	\end{enumerate}
	\item (2.28) Use the Polig-Hellman algorithm (Theorem 2.31) to solve 
		the discrete logarithm problem 
		\begin{displaymath}
			g^x = a \mod p 
		\end{displaymath}
		in each of the following cases. 
	\begin{enumerate}
		\item $p=433$, $g=7$, $a=166$
		\item $p=746497$, $g=10$, $a=243278$
		\item $p=41022299$, $g=2$, $a=39183497$ (Hint: $p = 2 \cdot 29^5 
			+ 1$.)
		\item $p=1291799$, $g=17$, $a=192988$ (Hint: $p-1$ has a factor 
			of $709$). 
	\end{enumerate}
\end{enumerate}
\end{document}
